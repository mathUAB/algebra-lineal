\documentclass[11pt,a4paper]{article}
\usepackage[utf8]{inputenc}
\usepackage{amsmath}
\usepackage{amsthm}
\usepackage{mathtools}
\usepackage{amssymb}
\usepackage[left=1.5cm,right=1.5cm,top=2cm,bottom=2cm]{geometry}

\newcommand*{\QED}{\hfill\ensuremath{\square}}
\renewcommand{\labelenumii}{(\alph{enumii})}
\renewcommand{\labelenumiii}{\roman{enumiii}.}
\newcommand{\rvline}{\hspace*{-\arraycolsep}\vline\hspace*{-\arraycolsep}}

\begin{document}
\begin{flushright}
    Víctor Ballester\par NIU:1570866\par Gener 2020
\end{flushright}
\rule{180mm}{0.1mm}
\begin{enumerate}
    \item Sigui $E$ l’$\mathbb{R}$-espai vectorial dels polinomis amb coeficients reals de grau menor o igual que, és a dir, $E=\{p(x)\in \mathbb{R}[x]\mid \text{gr}(p(x))\leq 3\}$. Siguin $$F=\{a_0+a_1x+a_2x^2+a_3x^3\in E\mid -a_0+a_1+a_3=0\}$$ $$G=\langle 1-x-x^2+x^3,2+x-x^2\rangle$$
    \begin{enumerate}
        \item Calculeu la dimensió i una base de $F$, $G$, $F\cap G$, $F+G$.\par 
        Per l'isomorfisme de coordenació tenim l'isomorfisme següent:
        \begin{align*}
            \mathbb{R}^4&\longrightarrow F\\
            (a_0,a_1,a_2,a_3)&\longmapsto a_0+a_1x+a_2x^2+a_3x^3
        \end{align*}
        Sabent, a més, que tots els polinomis de $F$ han de complir que $a_0=a_1+a_3$, llavors tenim que les coordenades de $F$ a $\mathbb{R}^4$ seran $(a_1+a_3,a_1,a_2,a_3)=a_1(1,1,0,0)+a_2(0,0,1,0)+a_3(1,0,0,1)$ i, per tant, una base de $F$ és $B_F=((1,1,0,0),(0,0,1,0),(1,0,0,1))$ i la dimensió de $F$ és $\text{dim}(F)=3$.\par 
        D'altra banda i de la mateixa manera que anteriorment, agafem les coordenades dels dos polinomis que generen G i vegem que són linealment independents:
        \begin{equation*}
            \begin{pmatrix}
              1 & -1 & -1 & 1 \\
              2 & 1 & -1 & 0 \\
           \end{pmatrix}
        \end{equation*}
        Clarament ho són, per la última coordenada nul·la del segon polinomi. Així una base de $G$ és $B_G=((1,-1,-1,1),(2,1,-1,0)$ i la dimensió de $G$ és $\text{dim}(G)=2$.\par 
        Pel que fa a la intersecció i la suma dels dos subespais, comencem posant a una matriu els vectors de les bases $B_F$ i $B_G$ amb la matriu identitat al costat.
        \begin{multline*}
            \begin{pmatrix}
            \begin{matrix}
            1 & 1 & 0 & 0 \\
            0 & 0 & 1 & 0 \\
            1 & 0 & 0 & 1 \\
            1 & -1 & -1 & 1 \\
            2 & 1 & -1 & 0 \\
            \end{matrix} & \rvline & \begin{matrix}
            1 & 0 & 0 & 0 & 0  \\
            0 & 1 & 0 & 0 & 0  \\
            0 & 0 & 1 & 0 & 0  \\
            0 & 0 & 0 & 1 & 0  \\
            0 & 0 & 0 & 0 & 1 \\
            \end{matrix}
            \end{pmatrix}\rightarrow\begin{pmatrix}
            \begin{matrix}
            1 & 1 & 0 & 0 \\
            0 & 0 & 1 & 0 \\
            0 & -1 & 0 & 1 \\
            0 & -2 & -1 & 1 \\
            0 & -1 & -1 & 0 \\
            \end{matrix} & \rvline & \begin{matrix}
            1 & 0 & 0 & 0 & 0  \\
            0 & 1 & 0 & 0 & 0  \\
            -1 & 0 & 1 & 0 & 0  \\
            -1 & 0 & 0 & 1 & 0  \\
            -2 & 0 & 0 & 0 & 1 \\
            \end{matrix}
            \end{pmatrix}\rightarrow \\ \rightarrow\begin{pmatrix}
            \begin{matrix}
            1 & 1 & 0 & 0 \\
            0 & 0 & 1 & 0 \\
            0 & -1 & 0 & 1 \\
            0 & 0 & -1 & -1 \\
            0 & 0 & -1 & -1 \\
            \end{matrix} & \rvline & \begin{matrix}
            1 & 0 & 0 & 0 & 0  \\
            0 & 1 & 0 & 0 & 0  \\
            -1 & 0 & 1 & 0 & 0  \\
            1 & 0 & -2 & 1 & 0  \\
            -1 & 0 & -1 & 0 & 1 \\
            \end{matrix}
            \end{pmatrix}\rightarrow\begin{pmatrix}
            \begin{matrix}
            1 & 1 & 0 & 0 \\
            0 & 0 & 1 & 0 \\
            0 & -1 & 0 & 1 \\
            0 & 0 & -1 & -1 \\
            0 & 0 & 0 & 0 \\
            \end{matrix} & \rvline & \begin{matrix}
            1 & 0 & 0 & 0 & 0  \\
            0 & 1 & 0 & 0 & 0  \\
            0 & 0 & 1 & 0 & 0  \\
            -1 & 0 & 0 & 1 & 0  \\
            -2 & 0 & 1 & -1 & 1 \\
            \end{matrix}
            \end{pmatrix}
        \end{multline*}
        Per tant, una base de $F+G$ és $B_{F+G}=((1,1,0,0),(0,0,1,0),(0,-1,0,1),(0,0,-1,-1))$ i, per tant, $\text{dim} (F+G)=4$. Pel que fa a la intersecció, sigui $\textbf{v}\in F\cap G$. Aleshores, $\textbf{v}\in F$ i $\textbf{v}\in G$. Per tant podem escriure el vector $\textbf{v}$ com a combinació lineal dels vectors de $B_{F}$ i $B_{G}$:
        \begin{align*}
            \textbf{v}&=a(1,1,0,0)+b(0,0,1,0)+c(1,0,0,1)\\
            \textbf{v}&=d(1,-1,-1,1)+e(2,1,-1,0)
        \end{align*} Igualant les dues expressions tenim que: $$a(1,1,0,0)+b(0,0,1,0)+c(1,0,0,1)+(-d)(1,-1,-1,1)+(-e)(2,1,-1,0)=0$$ Si ens fixem bé, els termes $a,b,c,-d,-e$ són exactament els coeficients de la matriu invertible en la fila de $0$'s de la matriu principal. Per tant, substituint tenim que $a=-2,b=0,c=1,d=1,e=-1$. Substituint aquests valors a una de les expressions anteriors de $\textbf{v}$ tenim que: $\textbf{v}=(-1,-2,0,1)$ i, per tant, aquest vector forma una base de $F\cap G$: $B_{F\cap G}=((-1,-2,0,1))$. Per la fórmula de Gra\ss mann tenim que $\text{dim}(F\cap G)=\text{dim}(F)+\text{dim}(G)-\text{dim}(F+G)=3+2-4=1$ i ens quadre amb el nombre de vectors de $B_{F\cap G}$.
        \item Amplieu la base de $F\cap G$ que heu trobat a una base de $F$ i a una base de $G$.\par
        Per ampliar la base de $F\cap G$ a una base de $F$ cal crear una matriu amb els vectors de $B_{F}$ i $B_{F\cap G}$ i estudiar la dependència lineal d'aquests.
        \begin{equation*}
            \begin{pmatrix}
            -1 & -2 & 0 & 1 \\
            1 & 1 & 0 & 0 \\
            0 & 0 & 1 & 0 \\
            1 & 0 & 0 & 1 \\
            \end{pmatrix}\rightarrow
            \begin{pmatrix}
            -1 & -2 & 0 & 1 \\
            0 & -1 & 0 & 1 \\
            0 & 0 & 1 & 0 \\
            0 & -2 & 0 & 2 \\
            \end{pmatrix} \rightarrow
            \begin{pmatrix}
            -1 & -2 & 0 & 1 \\
            0 & -1 & 0 & 1 \\
            0 & 0 & 1 & 0 \\
            0 & 0 & 0 & 0 \\
            \end{pmatrix}
        \end{equation*}
        Així, una base de $F$ a partir de l'ampliació de la base de $F\cap G$ és $B_{F'}=((-1,-2,0,1),(0,-1,0,1),\\(0,0,1,0))$. Procedint de manera anàloga per G, creem la matriu amb els vectors de la base de $F\cap G$ i els vectors de la base de $G$.
        \begin{equation*}
            \begin{pmatrix}
            -1 & -2 & 0 & 1 \\
            1 & -1 & -1 & 1 \\
            2 & 1 & -1 & 0 \\
            \end{pmatrix}\rightarrow
            \begin{pmatrix}
            -1 & -2 & 0 & 1 \\
            0 & -3 & -1 & 2 \\
            0 & -3 & -1 & 2 \\
            \end{pmatrix} \rightarrow
            \begin{pmatrix}
            -1 & -2 & 0 & 1 \\
            0 & -3 & -1 & 2 \\
            0 & 0 & 0 & 0 \\
            \end{pmatrix}
        \end{equation*}
        Així, una base de $G$ a partir de l'ampliació de la base de $F\cap G$ és $B_{G'}=((-1,-2,0,1),\\(0,-3,-1,2))$.
    \end{enumerate}
    \item Sigui $p$ un primer, i sigui $\mathbb{F}_p=\mathbb{Z}/p\mathbb{Z}$ el cos amb $p$ elements, i considerem l’espai vectorial format pels polinomis $\mathbb{F}_p[x]$.
    \begin{enumerate}
        \item  Donat un polinomi $f(x)\in\mathbb{F}_p[x]$, sigui $(f)=\{h(x)\in \mathbb{F}_p[x]\mid h(x) \text{ és múltiple de } f(x)\}$. Demostreu que $(f)$ és un subespai vectorial de $\mathbb{F}_p[x]$.\par Per demostrar que $(f)$ és subespai vectorial de $\mathbb{F}_p[x]$ cal veure tres coses:
        \begin{itemize}
            \item $0\in (f)$:\par
            Sabem, per definició, que qualsevol $h(x)\in (f)$ el podem escriure de la forma $h(x)=f(x)p(x)$ amb $p(x)\in\mathbb{F}_p[x]$. Així fent $p(x)=0$ tenim que $h(x)=0$ i $0\in (f)$.
            \item Donats $g(x),h(x)\in (f)$ aleshores $g(x)+h(x)\in (f)$:\par
            De la mateixa manera que anteriorment, podem escriure els polinomis $g(x)$ i $h(x)$ com $g(x)=f(x)p(x)$ i $h(x)=f(x)q(x)$ amb $p(x),q(x)\in\mathbb{F}_p[x]$. Sumant les equacions tenim que $g(x)+h(x)=f(x)p(x)+f(x)q(x)=f(x)(p(x)+q(x))$ i, per tant, $g(x)+h(x)\in (f)$.
            \item Donat un $g(x)\in (f)$ i un $\lambda(x)\in\mathbb{F}_p[x]$ aleshores $\lambda(x) g(x)\in (f)$:\par
            De la mateixa manera que anteriorment, tenim un $g(x)\in (f)$ que podem expressar de la forma $g(x)=f(x)p(x)$ amb $p(x)\in\mathbb{F}_p[x]$. Multiplicant l'expressió per un $\lambda(x)$ tenim que $\lambda(x)g(x)=\lambda(x)f(x)p(x)=f(x)(\lambda(x)p(x))$ i, per tant, $\lambda(x) g(x)\in (f)$.\par \QED
        \end{itemize}
        \item Sigui $L=\mathbb{F}_p[x]/(f)$ l'espai quocient. Trobeu la dimensió i una base de $L$. Quants elements té $L$? Comproveu que a $L$ el producte de classes definit a partir del producte de representants $([g]\cdot[h]=[gh])$ està ben definit.\par
        Com que tots els $h(x)\in (f)$, els múltiples de $f(x)$, tenen grau més gran o igual al grau de $f(x)$, és a dir, $\text{gr }h(x)\geq \text{gr }f(x)$, aleshores una base de $L$ estarà formada pels elements de la base de $\mathbb{F}_p[x]$ que no siguin a la base de $(f)$. Així com que els $h(x)$ tenen com a mínim el mateix grau que $f(x)$, una base de $(f)$ serà $B_{(f)}=(x^k,x^{k+1},x^{k+2},...)$ on $k=\text{gr }f(x)$. És clar que una base de $\mathbb{F}_p[x]$ és $B_{\mathbb{F}_p[x]}=(1,x,...,x^{k-1},x^k,x^{k+1},...)$. Així, mirant els elements que són a $B_{\mathbb{F}_p[x]}$ i no a $B_{(f)}$ seran els element de la base de $L$: $B_L=(1,x,...,x^{k-2},x^{k-1})$ i, per tant, $\text{dim }(L)=k$. Com que cada element de la base el podem multiplicar per $p$ nombres diferents i sabent que hi ha $k$ elements a la base tindrem que $L$ té $p^k$ elements.\par Per veure que el producte de classes definit a partir del producte de representants està ben definit, hem de veure que donades dues classes $[g]$, $[h]$ el seu producte és una classe de $L$. Així, efectuem el producte de dos representants $g$ i $h$, $q=gh$. Se'ns poden presentar dos casos: si $\text{gr}\;q(x)< \text{gr}\;f(x)$ o $\text{gr}\;q(x)\geq \text{gr}\;f(x)$. Pel primer cas és directe ja que $q$ és un representant de $L$ i, per tant, $[g]\cdot[h]=[q]$. Si se'ns dona el segon cas, hem de mirar el residu de la divisió euclídia de $q(x)$ per $f(x)$. De manera que $q(x)\equiv r(x)\mod f(x)$ per algun $r(x)\in L$. Sabem que $\text{gr}\;r(x)<\text{gr}\;f(x)$ i, per tant, $[g]\cdot[h]=[r]$.\par\QED
        
        
        \item Suposem ara que $f$ és irreductible a $\mathbb{F}_p[x]$ (és a dir, que no factoritza com a producte de dos polinomis de grau estrictament menor que el de $f$). Demostreu que la multiplicació de classes dona una estructura de cos a $L$. (Indicació: Per $[g]\neq 0$ considereu l’aplicació $\phi_{[g]}:L\longrightarrow L$ donada per $\phi_{[g]}([h])=[gh]$ i demostreu que és bijectiva).\par
        Perquè $L$ sigui un cos, la multiplicació de classes ha de complir les següents propietats:
        \begin{itemize}
            \item Propietat associativa:\par
            Siguin $[x],[y],[z]\in L$. Hem de demostrar que $([x]\cdot[y])\cdot[z]=[x]\cdot([y]\cdot[z])$.
            \begin{align*}
                ([x]\cdot[y])\cdot[z]&=[xy]\cdot[z]\qquad\text{ja que el producte de representants està ben definit a $L$.}\\
                &=[(xy)z]\\
                &=[x(yz)]\qquad\;\,\text{ja que la propietat associativa està ben definida a $\mathbb{F}_p[x]$.}\\
                &=[x]\cdot[yz]\\
                &=[x]\cdot([y]\cdot[z])
            \end{align*}
            \item Propietat commutativa:\par
            Siguin $[x],[y]\in L$. Hem de demostrar que $[x]\cdot[y]=[y]\cdot[x]$.
            \begin{align*}
                [x]\cdot[y]&=[xy]\qquad\text{ja que el producte de representants està ben definit a $L$.}\\
                &=[yx]\qquad\text{ja que la propietat commutativa està ben definida a $\mathbb{F}_p[x]$.}\\
                &=[y]\cdot[x]
            \end{align*}
            \item Existència de l'element neutre:\par
            Sigui $[p]\in L$. Llavors $\exists\, i\in L \mid [p]\cdot[i]=[i]\cdot[p]=[p]$. Trobem-lo:\par Aquest element $i$ és la classe dels polinomis $[i(x)]=1$, ja que $[i(x)]\cdot[p(x)]=[p(x)]\cdot[i(x)]=[p(x)]$ amb $[p(x)]\in L$. 
            \item Existència de l'element invers:\par
            Per aquesta propietat considerarem l'aplicació $\phi_{[g]}:L\longrightarrow L$ donada per $\phi_{[g]}([h])=[gh]$. Hem de demostrar que és bijectiva, és a dir, injectiva i exhaustiva. D'aquesta manera podrem definir la seva aplicació inversa i trobar l'element invers de $[g]$.
            \begin{itemize}
                \item Demostrem que és injectiva:\par
                Sigui $\phi_{[g]}([h])=\phi_{[g]}([k])$. Hem de veure que $[h]=[k]$.
                \begin{align*}
                    \phi_{[g]}([h])&=\phi_{[g]}([k])\\
                    [gh]&=[gk]\\
                    [g]\cdot[h]&=[g]\cdot[k]
                \end{align*}
                Com que $f$ és irreductible podem multiplicar $m-1$ vegades l'expressió anterior per $[g]$, on $m$ és l'ordre multiplicatiu de $[g]$ de manera que $[g]^m\equiv 1\bmod (f)$. Així, tenim que:
                \begin{align*}
                    [g]\cdot[h]&=[g]\cdot[k]\\
                    [g]^m\cdot[h]&=[g]^m\cdot[k]\\
                    [h]&=[k]
                \end{align*}
                com volíem demostrar.
                \item Demostrem que és exhaustiva:\par
                Sigui $[x]\in L$. Volem veure que existeix un $[h]$ tal que $\phi_{[g]}([h])=[x]$.
                \begin{align*}
                    \phi_{[g]}([h])&=[x]\\
                    [gh]&=[x]\\
                    [g]\cdot[h]&=[x]
                \end{align*}
                Com que $f$ és irreductible podem multiplicar $m-1$ vegades l'expressió anterior per $[g]$, on $m$ és l'ordre multiplicatiu de $[g]$ de manera que $[g]^m\equiv 1\bmod (f)$. Així, tenim que:
                \begin{align*}
                    [g]\cdot[h]&=[x]\\
                    [g]^m\cdot[h]&=[g]^{m-1}\cdot[x]\\
                    [h]&=[g]^{m-1}\cdot[x]
                \end{align*}
                i hem trobat un $[h]$ tal que $\phi_{[g]}([h])=[x]$.\par
            \end{itemize}
            Així, com que l'aplicació és bijectiva té sentit parlar de la inversa. Sigui $\varphi_{[k]}$ una aplicació tal que $\phi_{[g]}\circ\varphi_{[k]}=\varphi_{[k]}\circ\phi_{[g]}=id$. Aquesta aplicació $\varphi_{[k]}$ és la definida per $\varphi_{[k]}([h])=[kh]$. Fent la primera composició tenim que $\phi_{[g]}\circ\varphi_{[k]}([h])=[kgh]=[h]$ i fent la segona composició tenim que $\varphi_{[g]}\circ\phi_{[k]}([h])=[gkh]=[h]$. Observant els dos resultats veiem que $[kg]=[gk]=id$, és a dir, $[g]\cdot[k]=[k]\cdot[g]=id$ i, per tant, $[k]=[g]^{-1}$ és l'element invers de $[g]$ com volíem demostrar. 
            \item Propietat distributiva respecte la suma:\par
             Siguin $[x],[y],[z]\in L$. Hem de demostrar que $[x]\cdot([y]+[z])=[x]\cdot[y]+[x]\cdot[z]$. Com que de la suma de dos polinomis es realitza coeficient a coeficient, aquesta estarà ben definida a $L$. Així, demostrem que a $L$ es compleix la propietat distributiva del producte respecte la suma.
             \begin{align*}
                 [x]\cdot([y]+[z])&=[x]\cdot[y+z]\qquad \text{ja que la suma de representants està ben definida.}\\
                 &=[x(y+z)]\qquad\;\: \text{ja que el producte de representants està ben definit.}\\
                 &=[xy+xz]\qquad\;\; \text{ja que la propietat distributiva està ben definida a $\mathbb{F}_p[x]$.}\\
                 &=[xy]+[xz]\\
                 &=[x]\cdot[y]+[x]\cdot[z]
             \end{align*}
            \item $0\neq1$:\par
            Aquesta propietat és trivial de demostrar ja que el polinomi neutre per la suma no pot ser el mateix que el polinomi neutre per el producte, vegem-ho:\par
            Raonarem per reducció a l'absurd. Sigui $K$ un cos. Suposem que la suma i el producte tenen el mateix element neutre $i$. Per definició l'element neutre del producte es comporta de la manera següent: $i\cdot x=x\cdot i=x\quad\forall x\in K$. Però, a més, l'element neutre de la suma és absorbent en el producte, és a dir, $i\cdot x=x\cdot i=i\quad\forall x\in K$. Aleshores, estem dient que $i=x\quad\forall x \in K$, que és absurd. Per tant $0\neq 1$.
        \end{itemize}
        \QED
    \end{enumerate}
\end{enumerate}
\end{document}
