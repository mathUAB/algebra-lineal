\documentclass[11pt,a4paper]{article}
\usepackage[utf8]{inputenc}
\usepackage{amsmath}
\usepackage{amsthm}
\usepackage{mathtools}
\usepackage{amssymb}
\usepackage[left=1.5cm,right=1.5cm,top=2cm,bottom=2cm]{geometry}

\newcommand*{\QED}{\hfill\ensuremath{\square}}
\renewcommand{\labelenumii}{(\alph{enumii})}
\renewcommand{\labelenumiii}{(\roman{enumiii})}
\newcommand{\rvline}{\hspace*{-\arraycolsep}\vline\hspace*{-\arraycolsep}}

\begin{document}
\begin{flushright}
    Víctor Ballester\par NIU:1570866\par Maig 2020
\end{flushright}
\rule{180mm}{0.1mm}\par
Fixat $\alpha\in\mathbb{Q}$, sigui $f:M_2(\mathbb{Q})\rightarrow M_2(\mathbb{Q})$ l'aplicació definida per: 
$$f(A)=A\begin{pmatrix}
0 & \alpha\\
1 & 1
\end{pmatrix}-\begin{pmatrix}
0 & \alpha\\
1 & 1
\end{pmatrix}A$$ per a tota $A\in M_2(\mathbb{Q})$.
\begin{enumerate}
    \item Calculeu el polinomi característic de $f$.\par
    Sigui $A=\begin{pmatrix}
    a & b\\
    c & d
    \end{pmatrix}$ on $a,b,c,d\in\mathbb{Q}$. Aleshores, per la definició de $f$, tenim:
    \begin{multline*}
        f(A)=\begin{pmatrix}
    a & b\\
    c & d
    \end{pmatrix}\begin{pmatrix}
    0 & \alpha\\
    1 & 1
    \end{pmatrix}-\begin{pmatrix}
    0 & \alpha\\
    1 & 1
    \end{pmatrix}\begin{pmatrix}
    a & b\\
    c & d
    \end{pmatrix}=\begin{pmatrix}
    b & \alpha a+b\\
    d & \alpha c+d
    \end{pmatrix}-\begin{pmatrix}
    \alpha c & \alpha d\\
    a+c & b+d
    \end{pmatrix}=\\=\begin{pmatrix}
    b-\alpha c & \alpha a+b-\alpha d\\
    -a-c+d & -b+\alpha c
    \end{pmatrix}
    \end{multline*}
    Ara calculem la matriu de $f$ respecte la base canònica de $M_2(\mathbb{Q})$. Així obtenim:
    \begin{align*}
        f\left(\begin{pmatrix}
    1 & 0\\
    0 & 0
    \end{pmatrix}\right)=\begin{pmatrix}
    0 & \alpha\\
    -1 & 0
    \end{pmatrix}\\
    f\left(\begin{pmatrix}
    0 & 1\\
    0 & 0
    \end{pmatrix}\right)=\begin{pmatrix}
    1 & 1\\
    0 & -1
    \end{pmatrix}\\
    f\left(\begin{pmatrix}
    0 & 0\\
    1 & 0
    \end{pmatrix}\right)=\begin{pmatrix}
    -\alpha & 0\\
    -1 & \alpha
    \end{pmatrix}\\
    f\left(\begin{pmatrix}
    0 & 0\\
    0 & 1
    \end{pmatrix}\right)=\begin{pmatrix}
    0 & -\alpha\\
    1 & 0
    \end{pmatrix}
    \end{align*}
    D'aquesta manera podem obtenim fàcilment la matriu de $f$ respecte la base canònica:
    \begin{equation*}
        A=[f]_{Can(M_2(\mathbb{Q}))}=\begin{pmatrix}
    0 & 1 & -\alpha & 0\\
    \alpha & 1 & 0 & -\alpha\\
    -1 & 0 & -1 & 1\\
    0 & -1 & \alpha & 0
    \end{pmatrix}
    \end{equation*}
    Ara ja podem calcular el polinomi característic de $f$.
    \begin{multline*}
        p_f(x)=\text{det}(A-xI_4)=\begin{vmatrix}
        -x & 1 & -\alpha & 0\\
        \alpha & 1-x & 0 & -\alpha\\
        -1 & 0 & -1-x & 1\\
        0 & -1 & \alpha & -x
        \end{vmatrix}\stackrel{F_1\rightarrow F_1+F_4}{=}\begin{vmatrix}
        -x & 0 & 0 & -x\\
        \alpha & 1-x & 0 & -\alpha\\
        -1 & 0 & -1-x & 1\\
        0 & -1 & \alpha & -x
        \end{vmatrix}=\\=-x\begin{vmatrix}
        1 & 0 & 0 & 1\\
        \alpha & 1-x & 0 & -\alpha\\
        -1 & 0 & -1-x & 1\\
        0 & -1 & \alpha & -x
        \end{vmatrix}\stackrel{C_4\rightarrow C_4-F_1}{=}-x\begin{vmatrix}
        1 & 0 & 0 & 0\\
        \alpha & 1-x & 0 & -2\alpha\\
        -1 & 0 & -1-x & 2\\
        0 & -1 & \alpha & -x
        \end{vmatrix}=\\=-x\begin{vmatrix}
        1-x & 0 & -2\alpha\\
        0 & -1-x & 2\\
        -1 & \alpha & -x
        \end{vmatrix}=-x[x(1-x)(1+x)+2\alpha(1+x)-2\alpha(1-x)]=-x[x(1-x^2)+4\alpha x]=\\=x^2(x^2-4\alpha-1)
        \end{multline*}
        Així el polinomi característic de $f$ és $p_f(x)=x^2(x^2-4\alpha-1)=x^4-(4\alpha+1)x^2$.
    \item Per a quins valors de $\alpha$ és $f$ diagonalitzable?\par
    Primer de tot, observem que $f$ té un valor propi, el 0, amb multiplicitat algebraica 2 independent del valor de $\alpha$. Sabem que perquè $f$ sigui diagonalitzable, en particular, el polinomi característic de $f$ no pot tenir factors irreductibles de grau major a 1. Per tant, en el nostre cas, perquè $f$ sigui diagonalitzable hem de poder descompondre el polinomi $x^2-4\alpha-1$ en producte de dos polinomis de grau 1 amb coeficients racionals. Així, tenim que si es compleix $x^2-4\alpha-1=(x-z)(x+z)$ per algun $z\in\mathbb{Q}$ tindrem que $f$ diagonalitza. D'aquí és dedueix fàcilment que $z^2=4\alpha+1$, és a dir, $\alpha=\frac{z^2-1}{4}$. Observem que si $z=0$ ($\alpha=-1/4$), cal estudiar amb més detall el problema ja que aleshores $f$ tindrà un sol valor propi, el 0, amb multiplicitat algebraica 4. Ara bé, sabem que si $f$ serà diagonalitzable si $\text{dim(ker}(f-0id))=4$. Per tant, per aquest cas particular, és suficient estudiar $\text{ker}(f)$.
    \begin{equation*}
        \text{ker}(f)=\{(x,y,z,t)\in \mathbb{Q}|A\begin{pmatrix}
                x\\
                y\\
                z\\
                t
            \end{pmatrix}=\begin{pmatrix}
                0\\
                0\\
                0\\
                0
            \end{pmatrix}\}
    \end{equation*}
    Per calcular el $\text{ker}(f)$ solucionem el sistema homogeni:
    \begin{equation*}
            A\begin{pmatrix}
                x\\
                y\\
                z\\
                t
            \end{pmatrix}=\begin{pmatrix}
                0 & 1 & 1/4 & 0\\
                -1/4 & 1 & 0 & 1/4\\
                -1 & 0 & -1 & 1\\
                0 & -1 & -1/4 & 0
            \end{pmatrix}\begin{pmatrix}
                x\\
                y\\
                z\\
                t
            \end{pmatrix}=\begin{pmatrix}
                0\\
                0\\
                0\\
                0
            \end{pmatrix}
        \end{equation*}
        \begin{multline*}
            \begin{pmatrix}
                0 & 1 & 1/4 & 0 & \rvline & 0 \\
                -1/4 & 1 & 0 & 1/4 & \rvline & 0 \\
                -1 & 0 & -1 & 1 & \rvline & 0 \\
                 0 & -1 & -1/4 & 0 & \rvline & 0
            \end{pmatrix}\stackrel{F_4\rightarrow F_4+F_1}{\sim}\begin{pmatrix}
                0 & 1 & 1/4 & 0 & \rvline & 0 \\
                -1/4 & 1 & 0 & 1/4 & \rvline & 0 \\
                -1 & 0 & -1 & 1 & \rvline & 0 \\
                 0 & 0 & 0 & 0 & \rvline & 0
            \end{pmatrix}\stackrel{C_1\rightarrow C_1+C_4}{\sim}\\\sim\begin{pmatrix}
                0 & 1 & 1/4 & 0 & \rvline & 0 \\
                0 & 1 & 0 & 1/4 & \rvline & 0 \\
                0 & 0 & -1 & 1 & \rvline & 0 \\
                0 & 0 & 0 & 0 & \rvline & 0
            \end{pmatrix}\stackrel{F_2\rightarrow F_2-F_1}{\sim}\begin{pmatrix}
                0 & 1 & 1/4 & 0 & \rvline & 0 \\
                0 & 0 & -1/4 & 1/4 & \rvline & 0 \\
                0 & 0 & -1 & 1 & \rvline & 0 \\
                0 & 0 & 0 & 0 & \rvline & 0
            \end{pmatrix}\stackrel{F_3\rightarrow F_3-4F_2}{\sim}\\\sim\begin{pmatrix}
                0 & 1 & 1/4 & 0 & \rvline & 0 \\
                0 & 0 & -1/4 & 1/4 & \rvline & 0 \\
                0 & 0 & 0 & 0 & \rvline & 0 \\
                0 & 0 & 0 & 0 & \rvline & 0
            \end{pmatrix}\stackrel{F_1\rightarrow F_1+F_2}{\sim}\begin{pmatrix}
                0 & 1 & 0 & 1/4 & \rvline & 0 \\
                0 & 0 & -1/4 & 1/4 & \rvline & 0 \\
                0 & 0 & 0 & 0 & \rvline & 0 \\
                0 & 0 & 0 & 0 & \rvline & 0
            \end{pmatrix}\stackrel{F_2\rightarrow -4F_2}{\sim}\begin{pmatrix}
                0 & 1 & 0 & 1/4 & \rvline & 0 \\
                0 & 0 & 1 & -1 & \rvline & 0 \\
                0 & 0 & 0 & 0 & \rvline & 0 \\
                0 & 0 & 0 & 0 & \rvline & 0
            \end{pmatrix}
        \end{multline*}
        Per tant, tenim que les solucions generals al sistema d'equacions són $(x,y,z,t)=(x,-t/4,t,t)$ i per tant, $\text{ker}(f)=\langle(1,0,0,0),(0,-1,4,4)\rangle$. Observem directament que $\text{dim(ker}(f))=2\ne 4$ i, per tant, $f$ no diagonalitza quan $\alpha=-1/4\;(z=0)$. Per tant, tenim que els únics valors de $\alpha$ perquè $f$ diagonalitzi són expressions del tipus $\alpha=\frac{z^2-1}{4}\;\forall z\in\mathbb{Q},z\ne 0$.
\end{enumerate}
\end{document}
