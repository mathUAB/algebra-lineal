\documentclass[11pt,a4paper]{article}
\usepackage[utf8]{inputenc}
\usepackage{amsmath}
\usepackage{amsthm}
\usepackage{mathtools}
\usepackage{amssymb}
\usepackage[left=1.5cm,right=1.5cm,top=2cm,bottom=2cm]{geometry}

\newcommand*{\QED}{\hfill\ensuremath{\square}}
\renewcommand{\labelenumii}{(\alph{enumii})}
\renewcommand{\labelenumiii}{\roman{enumiii}.}
\newcommand{\rvline}{\hspace*{-\arraycolsep}\vline\hspace*{-\arraycolsep}}

\begin{document}
\begin{flushright}
    Víctor Ballester\par NIU:1570866\par Novembre 2019
\end{flushright}
\rule{180mm}{0.1mm}
\begin{enumerate}
    \item Siguin $$F_1=\langle(1,-1,1,-1),(2,-2,1,1),(-3,3,0,1),(0,0,5,6)\rangle $$ $$F_2=\langle(1,2,3,4),(-1,4,2,3)\rangle $$ subespais vectorials de $\mathbb{Q}^4$.
    \begin{enumerate}
        \item Calculeu la dimensió i una base dels subespais vectorials $F_1$, $F_2$, $F_1\cap F_2$ i $F_1+F_2$.\par
        Posem els vectors de $F_1$ en una matriu i calculem el rang d'aquesta matriu:
        \begin{equation*}
            \begin{pmatrix}
              1 & -1 & 1 & -1 \\
              2 & -2 & 1 & 1 \\
              -3 & 3 & 0 & 1 \\
              0 & 0 & 5 & 6 \\
           \end{pmatrix}\rightarrow\begin{pmatrix}
              1 & -1 & 1 & -1 \\
              0 & 0 & -1 & 3 \\
              0 & 0 & 3 & -2 \\
              0 & 0 & 5 & 6 \\
           \end{pmatrix}\rightarrow\begin{pmatrix}
              1 & -1 & 1 & -1 \\
              0 & 0 & -1 & 3 \\
              0 & 0 & 0 & 7 \\
              0 & 0 & 0 & 21 \\
           \end{pmatrix}
           \rightarrow\begin{pmatrix}
              1 & -1 & 1 & -1 \\
              0 & 0 & -1 & 3 \\
              0 & 0 & 0 & 7 \\
              0 & 0 & 0 & 0 \\
           \end{pmatrix}
        \end{equation*}
        Procedim de manera anàloga per $F_2$:
        \begin{equation*}
            \begin{pmatrix}
              1 & 2 & 3 & 4 \\
              -1 & 4 & 2 & 3 \\
           \end{pmatrix}\rightarrow\begin{pmatrix}
              1 & 2 & 3 & 4 \\
              0 & 6 & 5 & 7 \\
           \end{pmatrix}
        \end{equation*}
        Observant les files del les dues matrius esglaonades podem concloure que $\text{dim} (F_1)=3$ i $\text{dim} (F_2)=2$. Una base de $F_1$ és $B_{F_1}=((1,-1,1,-1),(0,0,-1,3),(0,0,0,7))$ i una base de $F_2$ és $B_{F_2}=((1,2,3,4),(0,6,5,7))$.\par Pel que fa a la intersecció i la suma dels dos subespais, comencem posant a una matriu els vectors de les bases $B_{F_1}$ i $B_{F_2}$ amb la matriu identitat al costat.
        \begin{multline*}
            \begin{pmatrix}
            \begin{matrix}
            1 & -1 & 1 & -1 \\
            0 & 0 & -1 & 3 \\
            0 & 0 & 0 & 7 \\
            1 & 2 & 3 & 4 \\
            0 & 6 & 5 & 7 \\
            \end{matrix} & \rvline & \begin{matrix}
            1 & 0 & 0 & 0 & 0  \\
            0 & 1 & 0 & 0 & 0  \\
            0 & 0 & 1 & 0 & 0  \\
            0 & 0 & 0 & 1 & 0  \\
            0 & 0 & 0 & 0 & 1 \\
            \end{matrix}
            \end{pmatrix}\rightarrow\begin{pmatrix}
            \begin{matrix}
            1 & -1 & 1 & -1 \\
            0 & 0 & -1 & 3 \\
            0 & 0 & 0 & 7 \\
            0 & 3 & 2 & 5 \\
            0 & 6 & 5 & 7 \\
            \end{matrix} & \rvline & \begin{matrix}
            1 & 0 & 0 & 0 & 0  \\
            0 & 1 & 0 & 0 & 0  \\
            0 & 0 & 1 & 0 & 0  \\
            -1 & 0 & 0 & 1 & 0  \\
            0 & 0 & 0 & 0 & 1 \\
            \end{matrix}
            \end{pmatrix}\rightarrow \\ \rightarrow\begin{pmatrix}
            \begin{matrix}
            1 & -1 & 1 & -1 \\
            0 & 0 & -1 & 3 \\
            0 & 0 & 0 & 7 \\
            0 & 3 & 2 & 5 \\
            0 & 0 & 1 & -3 \\
            \end{matrix} & \rvline & \begin{matrix}
            1 & 0 & 0 & 0 & 0  \\
            0 & 1 & 0 & 0 & 0  \\
            0 & 0 & 1 & 0 & 0  \\
            -1 & 0 & 0 & 1 & 0  \\
            2 & 0 & 0 & -2 & 1 \\
            \end{matrix}
            \end{pmatrix}\rightarrow\begin{pmatrix}
            \begin{matrix}
            1 & -1 & 1 & -1 \\
            0 & 0 & -1 & 3 \\
            0 & 0 & 0 & 7 \\
            0 & 3 & 2 & 5 \\
            0 & 0 & 0 & 0 \\
            \end{matrix} & \rvline & \begin{matrix}
            1 & 0 & 0 & 0 & 0  \\
            0 & 1 & 0 & 0 & 0  \\
            0 & 0 & 1 & 0 & 0  \\
            -1 & 0 & 0 & 1 & 0  \\
            2 & 1 & 0 & -2 & 1 \\
            \end{matrix}
            \end{pmatrix}
        \end{multline*}
        Per tant, una base de $F_1+F_2$ és $B_{F_1+F_2}=((1,-1,1,-1),(0,3,2,5),(0,0,-1,3),(0,0,0,7))$ i, per tant, $\text{dim} (F_1+F_2)=4$. Pel que fa a la intersecció ens fixem en la fila de $0$'s de la matriu principal i mirem els coeficients de la matriu invertible en aquesta fila. Per la intersecció, sigui $\textbf{v}\in F_1\cap F_2$. Aleshores, $\textbf{v}\in F_1$ i $\textbf{v}\in F_2$. Per tant podem escriure el vector $\textbf{v}$ com a combinació lineal dels vectors de $B_{F_1}$ i $B_{F_2}$:
        \begin{align*}
            \textbf{v}&=a(1,-1,1,-1)+b(0,0,-1,3)+c(0,0,0,7)\\
            \textbf{v}&=d(1,2,3,4)+e(0,6,5,7)
        \end{align*} Igualant les dues expressions tenim que: $$a(1,-1,1,-1)+b(0,0,-1,3)+c(0,0,0,7)+(-d)(1,2,3,4)+(-e)(0,6,5,7)=0$$ Si ens fixem bé, els termes $a,b,c,-d,-e$ són exactament els coeficients de la matriu invertible en la fila de $0$'s de la matriu principal. Per tant, substituint tenim que $a=2,b=1,c=0,d=2,e=1$. Substituint aquests valors a una de les expressions anteriors de $\textbf{v}$ tenim que: $\textbf{v}=(2,-2,1,1)$ i, per tant, aquest vector forma una base de $F_1\cap F_2$: $B_{F_1\cap F_2}=((2,-2,1,1))$. Per la fórmula de Gra\ss mann tenim que $\text{dim}(F_1\cap F_2)=\text{dim}(F_1)+\text{dim}(F_2)-\text{dim}(F_1+ F_2)=3+2-4=1$ i ens quadre amb el nombre de vectors de $B_{F_1\cap F_2}$.
        \item Amplieu la base de $F_1\cap F_2$ que heu trobat a una base de $F_1$ i a una base de $F_2$.\par
        Com que el vector de la base de $F_1\cap F_2$ és un vector contingut en la base de $F_1$, l'ampliació de la base de $F_1\cap F_2$ a una nova base de $F_1$ és exactament la base $B_{F_1}$ que hem trobat anteriorment. Per ampliar la base de $F_1\cap F_2$ a una base de $F_2$ cal crear una matriu amb els vectors de $B_{F_2}$ i $B_{F_1\cap F_2}$ i estudiar la dependència lineal d'aquests.
        \begin{equation*}
            \begin{pmatrix}
              2 & -2 & 1 & 1 \\
              1 & 2 & 3 & 4 \\
              0 & 6 & 5 & 7 \\
           \end{pmatrix}\rightarrow\begin{pmatrix}
              2 & -2 & 1 & 1 \\
              -2 & -4 & -6 & -8 \\
              0 & 6 & 5 & 7 \\
           \end{pmatrix}\rightarrow\begin{pmatrix}
              2 & -2 & 1 & 1 \\
              0 & -6 & -5 & -7 \\
              0 & 6 & 5 & 7 \\
           \end{pmatrix}\rightarrow\begin{pmatrix}
              2 & -2 & 1 & 1 \\
              0 & 0 & 0 & 0 \\
              0 & 6 & 5 & 7 \\
           \end{pmatrix}
        \end{equation*}
        Per tant, una base de $F_2$ ampliada a partir de la base de $F_1\cap F_2$ és: $B'_{F_2}=((2,-2,1,1),\\(0,6,5,7))$.
        
        \item Trobeu un sistema d’equacions lineal homogeni tal que $F_1$ sigui el conjunt de les solucions d’aquest sistema.\par
       Per tal de crear aquest sistema lineal homogeni, creem el sistema d'equacions que consisteix en posar els vectors  de $F_1$ com a columnes d'una matriu i afegir el vector $(x,y,z,t)$ com una columna de la matriu ampliada. Esglaonant la matriu fins obtenir una fila de zeros (l'obtindrem sigui com sigui ja que $\text{dim} (F_1)=3$) tenim que:
       \begin{equation*}
            \begin{pmatrix}\begin{matrix}
              1 & 2 & -3 & 0 \\
              -1 & -2 & 3 & 0 \\
              1 & 1 & 0 & 5 \\
              -1 & 1 & 1 & 6 \\
              \end{matrix} & \rvline & \begin{matrix}
              x\\
              y\\
              z\\
              t\\
              \end{matrix}
           \end{pmatrix}\rightarrow\begin{pmatrix}\begin{matrix}
              1 & 2 & -3 & 0 \\
              0 & 0 & 0 & 0 \\
              0 & -1 & 3 & 5 \\
              0 & 3 & -2 & 6 \\
              \end{matrix} & \rvline & \begin{matrix}
              x\\
              x+y\\
              z-x\\
              x+t\\
              \end{matrix}
           \end{pmatrix}
        \end{equation*}
        Per tant, observant l'element de la matriu ampliada de la fila de zeros de la matriu principal, veiem que $x+y=0$ és el sistema lineal homogeni que té com a solucions el subespai $F_1$. 
    \end{enumerate}
    
    
    
    \item En aquest exercici estudiem $\mathbb{R}$ com a $\mathbb{Q}$-espai vectorial.
    \begin{enumerate}
        \item Siguin $m_1,\ldots,m_k$ enters lliures de quadrats (si un primer $p$ divideix a $m_i$ aleshores $p^2$ no el divideix) i coprimers dos a dos (els primers que divideixen $m_i$ no divideixen $m_j$, $\forall\, i,j$). Demostreu per inducció que el conjunt $\{\sqrt{m_1},\ldots,\sqrt{m_k}\}$ és un conjunts d’elements $\mathbb{Q}$-linealment independents.\par
        Demostrarem una cosa més forta que el que ens demanen. Demostrarem que el conjunt $L_{k+1}=\{1,\sqrt{m_1},\ldots,\sqrt{m_{k+1}},\ldots,\sqrt{m_1m_2},\ldots,\sqrt{m_1\cdots m_{n+1}}\}$ és $\mathbb{Q}$-linealment independent. En concret el conjunt $L_{k+1}$ és el conjunt generat per 1, per les arrels dels primers k nombres donats lliures de quadrats i coprimers dos a dos i també per tots els possibles productes d'aquestes arrels entre elles combinades entre si (en total, $L_k$ té exactament $2^k$ elements). Per veure-ho demostrem primer que $L_k$ és un extensió del cos sobre el cos dels racionals, és a dir, que podem expressar qualsevol element $x\in L_k$ com $x=\alpha_1+\alpha_2\sqrt{m_1}+\cdots +\alpha_{2^k}\sqrt{m_1\cdots m_{k}}$. Per demostrar-ho, hem de veure que $L_k$ és tancat per producte i per inversos. Agafem dos elements $a,b\in L_k$. Hem de veure que $ab\in L_k$. Sabem que $a$ serà de la forma $a=\sqrt{p_1\cdots p_r}$ i $b$ serà de la forma $b=\sqrt{q_1\cdots q_s}$. El seu producte serà de la forma $ab=\sqrt{p_1\cdots p_r}\sqrt{q_1\cdots q_s}$. Ara, sigui $M$ el producte de tots els $p_i$ tals que $p_i=q_j$ per alguns $i,j$. Aleshores, $ab=M\sqrt{p_1\cdots p_{i-1}p_{i+1}\cdots p_r}\sqrt{q_1\cdots q_{j-1}q_{j+1}\cdots q_s}$ que clarament pertany a $L_k$. Ara hem de veure que donat un $c\in L_k$ $\exists \,d \mid cd=1$ i $d=c^{-1}$. Sabem que $c$ és de la forma $c=\sqrt{n_1\cdots n_t}$ i $d$ ha de ser de la forma $d=\frac{1}{\sqrt{n_1\cdots n_t}}=\frac{\sqrt{n_1\cdots n_t}}{n_1\cdots n_t}=\frac{1}{n_1\cdots n_t}\sqrt{n_1\cdots n_t}\in L_k$. Per tant, hem demostrat que $L_k$ és un cos. De la mateixa manera es demostra que tots els $L_i$ amb $1\leq i\leq k$ són cossos extensions del del cos $\mathbb{Q}$. De la definició de cos podem expressar qualsevol $x\in L_i$ com $x=a+b\sqrt{m_i}$ amb $a,b \in L_{i-1}$, és a dir, una base $B_{L_i}$ sobre $L_{i-1}$ és $B_{L_i}=(1,\sqrt{m_i})$, que té dimensió 2. De forma similar una base una base $B_{L_i}$ sobre $L_{i-2}$ és $B_{L_i}=(1,\sqrt{m_i},\sqrt{m_{i-1}},\sqrt{m_{i-1}m_i})$. En particular podrem expressar una base de $L_k$ sobre $\mathbb{Q}$ com $B_{L_k}=(1,\sqrt{m_1},\ldots,\sqrt{m_{n+1}},\ldots,\sqrt{m_1m_2},\ldots,\sqrt{m_1\cdots m_{n+1}})$ que tindrà $2^k$ elements. \par Amb aquesta informació introductòria, demostrarem per inducció sobre $k$ que el conjunt $L_{k+1}=\{1,\sqrt{m_1},\ldots,\sqrt{m_{k+1}},\ldots,\sqrt{m_1m_2},\ldots,\sqrt{m_1\cdots m_{n+1}}\}$ és linealment independent.\par Demostrem primer el cas k=1:\par Sigui $B_{L_1}=(1,\sqrt{m_1})$ una base de $L_1$ sobre $\mathbb{Q}$. Hem de demostrar que la base és  $\mathbb{Q}$-linealment independent. Suposem que no. Per tant aquesta base tindrà dimensió 1, és a dir, generarà el mateix espai que $\mathbb{Q}$, per tant, $\sqrt{m_1}\in \mathbb{Q}$. És a dir, podem expressar $\sqrt{m1}$ com $\sqrt{m_1}=\frac{x}{y}$ sent $x, y \in \mathbb{Z}^\ast$ i coprimers entre si. Això és impossible ja que sabem que $m_1$ és lliure de quadrats. Vegem-ho. Expressant $m_1$ com a producte de primers tenim que $m_1=r_1\cdots r_l$ i retocant l'expressió anterior, tenim que:
        \begin{align*}
            \frac{x}{y}&=\sqrt{m_1} \\
            \frac{x^2}{y^2}&=m_1\\
            x^2&=(r_1\cdots r_l)y^2
        \end{align*}
        Si $x=(r_1\cdots r_l)z$, tenim que:
        \begin{align*}
            x^2&=(r_1\cdots r_l)y^2\\
            z^2(r_1\cdots r_l)^2&=(r_1\cdots r_l)y^2\\
            z^2(r_1\cdots r_l)&=y^2
        \end{align*}
        Llavors $(r_1\cdots r_l)$ divideix $y^2$, i en particular divideix $y$. Però havíem dit que $x$ i $y$ eren coprimers, per tant, contradicció i $L_1$ és $\mathbb{Q}$-linealment independent.\par Ara suposem cert que $L_k$ és $\mathbb{Q}$-linealment independent i provem que $L_{k+1}$ també ho és. Per veure-ho, raonarem per reducció a l'absurd. En particular, si $L_{k+1}$ és linealment dependent la base $B_{L_{k+1}}$ sobre el cos $L_k$ no té dimensió 2, sinó que $\text{dim} (B_{L_{k+1}})=1$, de manera que els subespais $L_k$ i $L_{k+1}$ tenen la mateixa dimensió i, en conseqüència, $\sqrt{m_{k+1}}\in L_k$. Aleshores podrem expressar $\sqrt{m_{k+1}}$ de las següent manera:
        $$\sqrt{m_{k+1}}=\alpha + \beta \sqrt{m_k}$$ amb $\alpha,\beta \in L_{k-1}$. Retocant l'expressió, tenim que:
        \begin{align*}
            (\sqrt{m_{k+1}})^2&=(\alpha + \beta \sqrt{m_k})^2\\
            m_{k+1}&=\alpha^2+\beta^2m_k+2\alpha\beta\sqrt{m_k}
        \end{align*}
        Si $\alpha\beta\neq 0$ tindrem una expressió de $\sqrt{m_k}$ en termes de nombres racionals, és a dir, $\text{dim} (L_k)=1\neq 2$ sobre el cos $L_{k-1}$, cosa que contradiu que és impossible per la H.I.. Per tant, $\alpha\beta= 0$. Si $\alpha=0$, aleshores obtenim la següent expressió:
        \begin{align*}
            m_{k+1}&=\beta^2m_k \\
            m_{k+1}m_k&=\beta^2m_k^2 \\
            \sqrt{m_{k+1}m_k}&=\beta m_k \\
            \beta &=\frac{1}{m_k}\sqrt{m_{k+1}m_k}
        \end{align*}
        i llavors $\beta \notin L_{k-1}$ per ser $m_{k+1}m_k$ coprimers entre si i, per tant, no ser quadrat de cap racional. Si, d'altra banda, $\beta=0$, tenim la següent expressió:
        \begin{align*}
            m_{k+1}&=\alpha^2 \\
            \alpha &=\sqrt{m_{k+1}}
        \end{align*}
        i llavors $\alpha\notin L_{k-1}$ per H.I.. Observem que el cas $\alpha=\beta =0$ no es pot donar ja que sinó $\sqrt{m_{k+1}}=0$, que és absurd. Per tant, hem demostrat que el conjunt $L_{k+1}$ és $\mathbb{Q}$-linealment independent.\par \QED



        \item Demostreu que, per a tot $k\geq 0$, el conjunt $\{2,\sqrt{2},\sqrt[4]{2},\ldots,\sqrt[2^k]{2}\}$ és un conjunt $\mathbb{Q}$-linealment independent. Per fer-ho, podeu seguir els següents passos:
        \begin{enumerate}
            \item Demostreu que $\forall m\geq 1$ el polinomi $x^m-2$ és irreductible a $\mathbb{Q}[x]$, és a dir, que no es pot escriure com a producte de dos polinomis a $\mathbb{Q}[x]$ de grau $\leq k$. En particular, per a qualsevol $k\geq 0$ el polinomi $x^{2^k}-2$ és irreductible. (Indicació: trobeu primer les arrels complexes de $x^m-2$, després escriviu $x^m-2=g(x)h(x)$ i raoneu per què els termes constant de $g(x)$ i $h(x)$ no poden ser enters).\par
            Sigui $S=\{x_1,\ldots,x_m\}$ el conjunt de solucions del $P(x)=x^m-2$. Aquest polinomi el podem expressar també de la forma $x^m=z=2+0i=2$. Sabem que el mòdul de $z$ és $|z|=2$ i l'argument és igual a $\theta=\text{arg} (z)=2\pi (n-1)\;\forall n\in\mathbb{N}^*$. Els nombres complexos els podem expressar de la forma $z=|z|e^{\theta i}$, de manera que tindrem que $z=2e^{2\pi (n-1)i}$. Substituint $z$ a l'equació per $x^m$, tenim que: $x^m=2e^{2\pi (n-1)i}$ i, per tant, $x=\sqrt[m]{2}e^{\frac{2\pi (n-1)i}{m}}$. Donant els primers $m$ valors naturals a $n$ (és a dir, $n=1,\ldots,m$), tenim que el conjunt de solucions de l'equació principal són:
            \begin{align*}
                x_1&=\sqrt[m]{2}\\
                x_2&=\sqrt[m]{2}e^{\frac{2\pi i}{m}}\\
                &\;\:\vdots\\
                x_m&=\sqrt[m]{2}e^{\frac{2\pi (m-1)i}{m}}
            \end{align*}
            De manera que podem factoritzar el polinomi $P(x)$ i expressar-lo en funció de les seves arrels: $$P(x)=(x-\sqrt[m]{2}\lambda_1)(x-\sqrt[m]{2}\lambda_2)\cdots(x-\sqrt[m]{2}\lambda_m)$$ on $\lambda_j=e^{\frac{2\pi (j-1)i}{m}}$ per a $1\leq j\leq m$. D'altra banda també podem factoritzar el polinomi com a producte de dos polinomis de grau més petit que $m$. Així tenim que:
            \begin{align*}
                P(x)&=g(x)h(x)\\
                x^m-2&=(x-a)(b_{m-1}x^{m-1}+b_{m-2}x^{m-2}+\cdots+b_1x+b_0)
            \end{align*}
            Hem de veure que $a,b_0\notin \mathbb{Q}$ i, en particular, $a,b_0\notin \mathbb{Z}$. Per demostrar-ho, ho raonarem per reducció a l'absurd. Suposem que són $a,b_0\in \mathbb{Q}$. Aleshores compararem els termes constants de les dos factoritzacions del polinomi $P(x)$ que hem fet anteriorment, que han de ser necessàriament iguals. Tenim llavors que:
            \begin{equation*}
                P(x)=(x-a)(b_{m-1}x^{m-1}+\cdots+b_0)&=(x-\sqrt[m]{2}\lambda_1)(x-\sqrt[m]{2}\lambda_2)\cdots(x-\sqrt[m]{2}\lambda_m) 
            \end{equation*}
            i en particular:
            \begin{align*}
                -ab_0&=(-\sqrt[m]{2})^m\lambda_1\cdots\lambda_m \\
                -ab_0&=(-\sqrt[m]{2})^me^0e^\frac{2\pi i}{m}\cdots e^\frac{2\pi (m-2)i}{m}e^\frac{2\pi (m-1)i}{m} \\
                -ab_0&=(-1)^m2e^0e^{2\pi i}\cdots e^{2\pi (m-2)i}e^{2\pi (m-1)i} \\
                (-1)^{m-1}\frac{ab_0}{2}&=e^{0+2\pi i+ 2\pi\cdot 2i+\cdots +2\pi(m-2)i+2\pi(m-1)i}\\
                (-1)^{m-1}\frac{ab_0}{2}&=e^{2\pi i(1+\cdots+m-1)}\\
                (-1)^{m-1}\frac{ab_0}{2}&=e^{2\pi i(\frac{m(m-1)}{2})}\\
                (-1)^{m-1}\frac{ab_0}{2}&=e^{\pi m(m-1)i}
            \end{align*}
            Clarament la part de l'esquerra és racional, mentre que la de la dreta és irracional, igualtat que és absurda, per tant, la afirmació inicial era falsa i $a,b_0\notin \mathbb{Q}$. En particular $a,b_0\notin \mathbb{Z}$. Hem demostrat que el polinomi $P(x)=x^m-2$ és irreductible a $\mathbb{Q}[x]$. En particular, fent $m=2^k$ per a qualsevol $k\geq 0$ tenim que el polinomi $P'(x)=x^{2^k}-2$ és irreductible.
            
            
            
            \item Demostreu que una combinació lineal no trivial dels elements fins a $\sqrt[2^k]{2}$ dona lloc a un polinomi de grau dividint $2^{k-1}$ que té $\sqrt[2^k]{2}$ com a arrel.\par
            Sabem que podem expressar els elements del conjunt $\{2,\sqrt{2},\sqrt[4]{2},\ldots,\sqrt[2^k]{2}\}$ com una combinació lineal no trivial d'aquests:
            \begin{equation*}
                2\mu_0+\sqrt{2}\mu_1+\cdots+\sqrt[2^j]{2}\mu_j+\cdots+\sqrt[2^k]{2}\mu_k=0
            \end{equation*}
            amb $\mu_0,\ldots,\mu_k$ no tots 0. Ara bé, fent $x=\sqrt[2^k]{2}$ i substituint tots els termes irracionals dels sumands anteriors per l'expressió de $x$, podem escriure el següent polinomi:
            \begin{equation*}
                Q(x)=\mu_1x^{2^{k-1}}+\cdots+\mu_jx^{2^{k-j}}+\cdots+\mu_kx+2\mu_0=0
            \end{equation*}
            que s'anul·la en $x=\sqrt[2^k]{2}$ i, per tant, $\sqrt[2^k]{2}$ és una arrel de $Q(x)$.
            \item Deduïu una contradicció fent servir els dos apartats anteriors (recordeu la divisió de polinomis: donats polinomis $a(x)$ i $b(x)$, existeixen polinomis $q(x)$ i $r(x)$ amb $\text{grau}(r(x))\leq \text{grau}(b(x))$ tals que $a(x) = b(x)q(x) + r(x)$.)\par
        \end{enumerate}
    \end{enumerate}
\end{enumerate}

\end{document}
