\documentclass[11pt,a4paper]{article}
\usepackage[utf8]{inputenc}
\usepackage{amsmath}
\usepackage{amsthm}
\usepackage{mathtools}
\usepackage{amssymb}
\usepackage[left=1.5cm,right=1.5cm,top=2cm,bottom=2cm]{geometry}

\newcommand*{\QED}{\hfill\ensuremath{\square}}
\renewcommand{\labelenumii}{(\alph{enumii})}
\renewcommand{\labelenumiii}{\roman{enumiii}.}
\newcommand{\rvline}{\hspace*{-\arraycolsep}\vline\hspace*{-\arraycolsep}}

\begin{document}
\begin{flushright}
    Víctor Ballester\par NIU:1570866\par Març 2020
\end{flushright}
\rule{180mm}{0.1mm}\par
\begin{enumerate}
    \item Sigui $f:E\longrightarrow E$ una aplicació lineal satisfent $f^2=f$.
    \begin{enumerate}
        \item Demostreu que, si $f$ és invertible, aleshores $f=\text{id}_E$.\par
        Com que $f$ és invertible, existeix una aplicació $g:E\longrightarrow E$ tal que $f\circ g=g\circ f=\text{id}_E$. Llavors tenim que: 
        \begin{align*}
            id_E&=f\circ g \qquad\qquad\;\text{Per definició de la inversa.}\\
            &=f^2\circ g \qquad\qquad\!\text{Per definició.}\\
            &=f\circ(f\circ g) \qquad\text{Per la propietat associativa}\\
            &=f\circ id_E \qquad\quad\;\text{Per definició de la inversa.}\\
            &=f
        \end{align*}\QED
        \item Demostreu que $E=\text{Ker }f\oplus\text{Imf }f$.\par
        Sigui $v\in E$. Aquest vector $v$ el podem escriure de la següent forma: $v=f(v)+(v-f(v))$. Veiem clarament que el primer terme de la dreta de l'expressió ($f(v)$) pertany a la imatge de $f$. Hem de veure que l'altre terme ($v-f(v)$) pertany al nucli. Definim $u=v-f(v)$ i vegem que $u\in \text{Ker }f$.
        \begin{align*}
            u&=v-f(v)\\
            f(u)&=f(v-f(v))\\
            f(u)&=f(v)-f^2(v)\\
            f(u)&=f(v)-f(v)\\
            f(u)&=0
        \end{align*}
        I, per tant, $u=v-f(v)\in \text{Ker }f$. Així, hem vist que $E=\text{Ker }f+\text{Imf }f$. Hem de veure ara que la suma és directa, és a dir, $\text{Ker }f\cap\text{Imf }f={0}$. Ho farem agafant un vector de la intersecció i veient que és el vector zero. Sigui $\omega\in\text{Ker }f\cap\text{Imf }f$. En particular com que $\omega\in\text{Ker }f$ tenim que $f(\omega)=0$. A més, com que $\omega\in\text{Im }f$, $\exists\,v \mid f(v)=\omega$. Ara bé,
        \begin{align*}
            f(v)&=\omega\\
            f^2(v)&=f(\omega)\\
            f^2(v)&=0\\
            f(v)&=0
        \end{align*}
        I, per tant, comparant la primera i la última equació obtenim $\omega=0$ com volíem. Finalment, $E=\text{Ker }f\oplus\text{Im }f$.\QED
        \item Demostreu que, si $E$ és de dimensió finita, aleshores existeix una base $\mathcal{B}$ de $E$ tal que 
        $$[f]_\mathcal{B}=
        \begin{pmatrix}
        I_k & 0\\
        0 & 0
        \end{pmatrix}$$ on $k=\text{dim}(\text{Im }f)$.\par
        Sigui $\mathcal{B},\mathcal{B'}$ dues bases de $E$ i siguin $[f]_\mathcal{B},[f]_\mathcal{B'}$ les matrius associades a l'aplicació d'aquestes bases. Sabem que $\text{rang }[f]_\mathcal{B}=\text{rang }[f]_\mathcal{B'}=\text{dim}(\text{Im }f)$. A més, pel teorema de la PAQ-reducció existeixen matrius invertibles $P,Q$ tals que $P[f]_\mathcal{B'}Q=\begin{pmatrix}
        I_k & 0\\
        0 & 0
        \end{pmatrix}$. Si definim $P$ com la matriu canvi de base de $\mathcal{B'}$ a $\mathcal{B}$, és a dir, $P=[\text{id}]_{\mathcal{B'},\mathcal{B}}$  i $Q$ com la matriu canvi de base de $\mathcal{B}$ a $\mathcal{B'}$, és a dir, $Q=[\text{id}]_{\mathcal{B},\mathcal{B'}}$ tenim que $P[f]_\mathcal{B'}Q=[f]_\mathcal{B}=\begin{pmatrix}
        I_k & 0\\
        0 & 0
        \end{pmatrix}$. Per tant, hem trobat una base $\mathcal{B}$ tal que a partir d'un base qualsevol $\mathcal{B'}$ i fent els canvis de base corresponents obtenim que la matriu associada a $f$ en la base $\mathcal{B}$ és $\begin{pmatrix}
        I_k & 0\\
        0 & 0
        \end{pmatrix}$. 
    \end{enumerate}
    
    \item 
    \begin{enumerate}
        \item Doneu, per a tot $n\geq 2$, un exemple d'aplicació $f_n:\mathbb{R}^n\longrightarrow \mathbb{R}^n$  que satisfaci $f_n^2=f_n$, amb $f_n\ne \text{id}_{\mathbb{R}^n}$,$0_{\mathbb{R}^n}$.\par
        Per a $n\geq 2$ considerem l'aplicació $f_n:\mathbb{R}^n\longrightarrow \mathbb{R}^n$ definida per $f_n(v_1,\ldots,v_n)=(\frac{v_1+\ldots+v_n}{n},\ldots,\\\frac{v_1+\ldots+v_n}{n})$.
        \item Podeu trobar per cada $2\leq k\leq n$ una aplicació $f_{k,n}:\mathbb{R}^n\longrightarrow \mathbb{R}^n$ diferent de la identitat tal que $f_{k,n}^k=f_{k,n}$ amb $k$ la potència mínima? (És a dir, tal que $f_{k,n}^l\ne f_{k,n}$ per a tot $l<k$?\par
        Agafem una $f_{k,n}$ tal que vagi permutant $k-1$ components d'un vector de $\mathbb{R}^n$, és a dir, definim la funció $f_{k,n}$ de la següent manera:
        \begin{align*}
            f_{k,n}:\mathbb{R}^n&\longrightarrow\mathbb{R}^n\\
            (\alpha_1,\ldots,\alpha_n)&\longmapsto(\alpha_{k-1},\alpha_1\ldots,\alpha_{k-2},\alpha_k,\ldots,\alpha_n)
        \end{align*}
        Veiem clarament que per $n=2$, obtindríem la identitat i, per tant, la funció $f_{k,n}$ només és vàlida per $n>2$. Per $n=2$ podem agafar la funció definida a l'apartat anterior, $f_2:\mathbb{R}^2\longrightarrow\mathbb{R}^2$ definida per $f_2(x,y)=(\frac{x+y}{2},\frac{x+y}{2})$. Per l'únic valor possible de $k$, $k=2$, aquesta funció verifica que $f_2^2=f_2$.
        \end{enumerate}
\end{enumerate}
\end{document}
