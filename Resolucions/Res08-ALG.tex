\documentclass[11pt,a4paper]{article}
\usepackage[utf8]{inputenc}
\usepackage{amsmath}
\usepackage{amsthm}
\usepackage{mathtools}
\usepackage{amssymb}
\usepackage[left=1.5cm,right=1.5cm,top=2cm,bottom=2cm]{geometry}

\newcommand*{\QED}{\hfill\ensuremath{\square}}
\renewcommand{\labelenumii}{(\alph{enumii})}
\renewcommand{\labelenumiii}{(\roman{enumiii})}
\newcommand{\rvline}{\hspace*{-\arraycolsep}\vline\hspace*{-\arraycolsep}}

\begin{document}
\begin{flushright}
    Víctor Ballester\par NIU:1570866\par Maig 2020
\end{flushright}
\rule{180mm}{0.1mm}\par
\begin{enumerate}
    \item Trobeu la forma canònica de Jordan de les matrius a coeficients reals:
    \begin{equation*}
        \begin{pmatrix}
        0 & -1 & -1 & -1\\
        1 & 2 & 1 & 1\\
        0 & 0 & 0 & -1\\
        0 & 0 & 1 & 2
        \end{pmatrix}\qquad\begin{pmatrix}
        0 & 1 & -2 & 1\\
        -2 & 1 & -6 & 3\\
        2 & -3 & 0 & 1\\
        2 & -3 & -2 & 3
        \end{pmatrix}
    \end{equation*}
    Canvia la resposta si les matrius les considerem a coeficients complexos?
\end{enumerate}
Anomenem $A$ a la primera matriu i $B$ a la segona.\par
Comencem calculant el polinomi característic de $A$. Per fer-ho observem que la matriu $A-xI_4$ és una matriu per blocs de la forma $\begin{pmatrix}
    M_1 & \rvline & M_2\\
    \hline
    0 & \rvline &  M_3
    \end{pmatrix}$, per a certes matrius $M_1,M_2,M_3\in\mathcal{M}_2(\mathbb{R})$. Ara bé, el determinant d'aquests tipus de matrius sabem que és igual a $\text{det}(M_1)\text{det}(M_3)$. Però si, a més, observem que $M_1=M_3=\begin{pmatrix}
    -x & -1\\
    1 &  2-x
    \end{pmatrix}$, obtenim:
    \begin{equation*}
        p_A(x)=\text{det}(A-xI_4)=\text{det}\left(\begin{pmatrix}
    -x & -1\\
    1 &  2-x
    \end{pmatrix}\right)^2=\begin{vmatrix}
        -x & -1\\
        1 &  2-x
        \end{vmatrix}^2=(x^2-2x+1)^2=(x-1)^4
    \end{equation*}
        Així, el polinomi característic de $A$ és $p_A(x)=(x-1)^4=x^4-4x^3+6x^2-4x+1$.
        Per calcular el polinomi mínim, observem que:
        \begin{equation*}
            (A-I_4)^2=\begin{pmatrix}
            -1 & -1 & -1 & -1\\
            1 & 1 & 1 & 1\\
            0 & 0 & -1 & -1\\
            0 & 0 & 1 & 1
            \end{pmatrix}^2=\begin{pmatrix}
            0 & 0 & 0 & 0\\
            0 & 0 & 0 & 0\\
            0 & 0 & 0 & 0\\
            0 & 0 & 0 & 0
            \end{pmatrix}
        \end{equation*}
        I, per tant, tenim que $(A-I_4)^2=0_4$, sent $0_4$ la matriu nul·la de mida $4\times4$. Com que, clarament, $A-I_4\ne 0_4$ i, pel fet que només hi ha el factor ($x-1$, amb multiplicitat 4) al polinomi característic obtenim que $m_A(x)=(x-1)^2$. Ara bé, com que el polinomi mínim té exponent 2 en el factor lineal de valor propi 1, el bloc més gran de la matriu de Jordan d'aquest valor propi serà de dimensió 2. Però com que, pel polinomi característic, veiem que el valor propi 1 té multiplicitat algebraica 4 obtenim dos possibles formes de Jordan:
        \begin{equation*}
            \begin{pmatrix}
            1 & 0 & 0 & 0\\
            1 & 1 & 0 & 0\\
            0 & 0 & 1 & 0\\
            0 & 0 & 1 & 1
            \end{pmatrix}\quad\text{ó}\quad\begin{pmatrix}
            1 & 0 & 0 & 0\\
            0 & 1 & 0 & 0\\
            0 & 0 & 1 & 0\\
            0 & 0 & 1 & 1
            \end{pmatrix}
        \end{equation*}
    Per saber quina d'aquestes dues és la correcta estudiarem el $\text{ker}(A-I_4)$. Si obtenim $\text{dim(ker}(A-I_4))=2$, clarament serà la primera opció, en canvi si obtenim $\text{dim(ker}(A-I_4))=3$ serà la segona opció.
    \begin{equation*}
        \text{ker}(A-I_4)=\{(x,y,z,t)\in \mathbb{R}\mid(A-I_4)\begin{pmatrix}
                x\\
                y\\
                z\\
                t
            \end{pmatrix}=\begin{pmatrix}
                0\\
                0\\
                0\\
                0
            \end{pmatrix}\}
    \end{equation*}
    Per calcular el $\text{ker}(A-I_4)$ solucionem el sistema homogeni:
    \begin{equation*}
            (A-I_4)\begin{pmatrix}
                x\\
                y\\
                z\\
                t
            \end{pmatrix}=\begin{pmatrix}
            -1 & -1 & -1 & -1\\
            1 & 1 & 1 & 1\\
            0 & 0 & -1 & -1\\
            0 & 0 & 1 & 1
            \end{pmatrix}\begin{pmatrix}
                x\\
                y\\
                z\\
                t
            \end{pmatrix}=\begin{pmatrix}
                0\\
                0\\
                0\\
                0
            \end{pmatrix}
        \end{equation*}
        \begin{multline*}
                \begin{pmatrix}
                -1 & -1 & -1 & -1 & \rvline & 0\\
                1 & 1 & 1 & 1 & \rvline & 0\\
                0 & 0 & -1 & -1 & \rvline & 0\\
                0 & 0 & 1 & 1 & \rvline & 0
                \end{pmatrix}\stackrel{F_2\rightarrow F_2+F_1}{\sim}\begin{pmatrix}
                -1 & -1 & -1 & -1 & \rvline & 0\\
                0 & 0 & 0 & 0 & \rvline & 0\\
                0 & 0 & -1 & -1 & \rvline & 0\\
                0 & 0 & 1 & 1 & \rvline & 0
            \end{pmatrix}\stackrel{F_4\rightarrow F_4+F_3}{\sim}\\\sim\begin{pmatrix}
                -1 & -1 & -1 & -1 & \rvline & 0\\
                0 & 0 & 0 & 0 & \rvline & 0\\
                0 & 0 & -1 & -1 & \rvline & 0\\
                0 & 0 & 0 & 0 & \rvline & 0
            \end{pmatrix}\stackrel{F_1\rightarrow F_1-F_3}{\sim}\begin{pmatrix}
                -1 & -1 & 0 & 0 & \rvline & 0\\
                0 & 0 & 0 & 0 & \rvline & 0\\
                0 & 0 & -1 & -1 & \rvline & 0\\
                0 & 0 & 0 & 0 & \rvline & 0
            \end{pmatrix}
        \end{multline*}
        Per tant, tenim que les solucions generals al sistema d'equacions són $(x,y,z,t)=(-y,y,-t,t)$ i, per tant, $\text{ker}(A-I_4)=\langle(-1,1,0,0),(0,0,-1,1)\rangle$. Observem directament que $\text{dim(ker}(A-I_4))=2$ i, per tant, la forma canònica de Jordan de $A$ ha de ser:
        \begin{equation*}
            J_A=\begin{pmatrix}
            1 & 0 & 0 & 0\\
            1 & 1 & 0 & 0\\
            0 & 0 & 1 & 0\\
            0 & 0 & 1 & 1
            \end{pmatrix}
        \end{equation*}
        Passem ara a estudiar la forma canònica de Jordan de la matriu $B$. Comencem primer calculant el polinomi característic de $B$.
        \begin{multline*}
        p_B(x)=\text{det}(B-xI_4)=\begin{vmatrix}
        -x & 1 & -2 & 1\\
        -2 & 1-x & -6 & 3\\
        2 & -3 & -x & 1\\
        2 & -3 & -2 & 3-x
        \end{vmatrix}\stackrel{F_4\rightarrow F_4-F_3}{=}\begin{vmatrix}
        -x & 1 & -2 & 1\\
        -2 & 1-x & -6 & 3\\
        2 & -3 & -x & 1\\
        0 & 0 & -2+x & 2-x
        \end{vmatrix}\stackrel{F_2\rightarrow F_2+F_3}{=}\\=(x-2)\begin{vmatrix}
        -x & 1 & -2 & 1\\
        0 & -2-x & -6-x & 4\\
        2 & -3 & -x & 1\\
        0 & 0 & 1 & -1
        \end{vmatrix}\stackrel{C_3\rightarrow C_3+C_4}{=}(x-2)\begin{vmatrix}
        -x & 1 & -1 & 1\\
        0 & -2-x & -2-x & 4\\
        2 & -3 & 1-x & 1\\
        0 & 0 & 0 & -1
        \end{vmatrix}=\\=-(x-2)\begin{vmatrix}
        -x & 1 & -1\\
        0 & -2-x & -2-x\\
        2 & -3 & 1-x\\
        \end{vmatrix}=(x-2)(x+2)\begin{vmatrix}
        -x & 1 & -1\\
        0 & 1 & 1\\
        2 & -3 & 1-x\\
        \end{vmatrix}\stackrel{F_1\rightarrow F_1-F_2}{=}\\=(x-2)(x+2)\begin{vmatrix}
        -x & 0 & -2\\
        0 & 1 & 1\\
        2 & -3 & 1-x\\
        \end{vmatrix}\stackrel{F_3\rightarrow F_3+3F_2}{=}(x-2)(x+2)\begin{vmatrix}
        -x & 0 & -2\\
        0 & 1 & 1\\
        2 & 0 & 4-x\\
        \end{vmatrix}=(x-2)(x+2)(x^2-4x+4)=\\=(x+2)(x-2)^3
        \end{multline*}
        Així, el polinomi característic de $B$ és $p_B(x)=(x+2)(x-2)^3=x^4-4x^3+16x-16$. Observem que, com que $m_B\mid p_B$, aleshores el polinomi mínim té tots els factors lineals diferents del polinomi característic amb, com a mínim, multiplicitat 1. És suficient doncs, per calcular el polinomi mínim, trobar el mínim $\alpha\in\{1,2,3\}$ tal que $m_B(A)=(A+2I_4)(A-2I_4)^\alpha=0_4$, on $0_4$ és la matriu nul·la de mida $4\times4$. Així, fent el producte $(A+2I_4)(A-2I_4)$ obtenim:
        \begin{equation*}
            (A+2I_4)(A-2I_4)=\begin{pmatrix}
        2 & 1 & -2 & 1\\
        -2 & 3 & -6 & 3\\
        2 & -3 & 2 & 1\\
        2 & -3 & -2 & 5
        \end{pmatrix}\begin{pmatrix}
        -2 & 1 & -2 & 1\\
        -2 & -1 & -6 & 3\\
        2 & -3 & -2 & 1\\
        2 & -3 & -2 & 1
        \end{pmatrix}=\begin{pmatrix}
        -8 & 4 & -8 & 4\\
        -8 & 4 & -8 & 4\\
        8 & -4 & 8 & -4\\
        8 & -4 & 8 & -4
        \end{pmatrix}
        \end{equation*}
        Com que la matriu resultant no és la matriu nul·la, multipliquem aquesta matriu un altre cop per la dreta per $A-2I_4$. Així doncs, obtenim:
        \begin{multline*}
            (A+2I_4)(A-2I_4)^2=[(A+2I_4)(A-2I_4)](A-2I_4)=\begin{pmatrix}
        -8 & 4 & -8 & 4\\
        -8 & 4 & -8 & 4\\
        8 & -4 & 8 & -4\\
        8 & -4 & 8 & -4
        \end{pmatrix}\begin{pmatrix}
        -2 & 1 & -2 & 1\\
        -2 & -1 & -6 & 3\\
        2 & -3 & -2 & 1\\
        2 & -3 & -2 & 1
        \end{pmatrix}=\\=\begin{pmatrix}
            0 & 0 & 0 & 0\\
            0 & 0 & 0 & 0\\
            0 & 0 & 0 & 0\\
            0 & 0 & 0 & 0
        \end{pmatrix}
        \end{multline*}
        que és la matriu nul·la. Per tant, el polinomi mínim de $B$ és $m_B(x)=(x+2I_4)(x-2I_4)^2$. En aquest cas, ja tenim suficient informació amb els polinomis característic i mínim per calcular la forma canònica de Jordan sense haver de calcular els kernels. Així doncs, com que el polinomi mínim té exponent 2 en el factor lineal de valor propi 2, el bloc més gran de la matriu de Jordan d'aquest valor propi serà de dimensió 2. De manera similar veiem que el bloc més gran de la matriu de Jordan de valor propi $-2$ serà de dimensió 1. Així doncs, només ens queda una única possibilitat (pel que fa a l'estructura de blocs):
        \begin{equation*}
            J_B=\begin{pmatrix}
            -2 & 0 & 0 & 0\\
            0 & 2 & 0 & 0\\
            0 & 0 & 2 & 0\\
            0 & 0 & 1 & 2
            \end{pmatrix}
        \end{equation*}
        Observem que si les matrius les considerem a coeficients complexos no canviarà res, ja que, el polinomi característic d'ambdues matrius ja ha factoritzat en productes de factors lineals al cos $\mathbb{R}$ i, per tant, de la inclusió $\mathbb{R}\subset\mathbb{C}$, tenim que el polinomi característic d'ambdues matrius també factoritzaria d'igual forma al cos $\mathbb{C}$ i, en conseqüència, les matrius de Jordan d'ambdues matrius serien idèntiques a les ja calculades.
\end{document}
